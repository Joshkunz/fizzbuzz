% Language: Plain TeX
% To Run:
\noindent
Like Vim, \TeX\ isn't really a conventional programming language. It's
a language for typesetting documents, that has some programming features
built in. The \TeX\ program generates ``dvi'' files that can be rendered into
a number of formats, the most common being pdf. To generate a pdf with
FizzBuzz from 1 to 100 run:

    {\tt \$ tex fizzbuzz.tex \par     % $ tex fizzbuzz.tex
         \$ dvipdf fizzbuzz.dvi }     % $ dvipdf fizzbuzz.tex

\noindent
Which should generate a {\tt fizzbuzz.pdf} file for you to view. If you're viewing
the pdf of this document, below is the output of fizzbuzz from 1 to 100:
% \fizzbuzz{1}{100}

% Everything Below this point is the implementation.

\newcount\fstart
\newcount\fend
\newcount\divis
\newcount\modq
\newif\ifdthree
\newif\ifdfive

% \mod{a}{n} will set the counter 'a' to the remainder
% of a / n
\def\mod #1#2{
    \modq=#1
    \divide\modq by #2
    \multiply\modq by #2
    \multiply\modq by -1
    \advance#1 by \modq
}

% \fb{\ctr} will output the fizzbuzz string for the 
% number in the counter
\def\fb #1{
    \number#1 
    \dthreefalse
    \dfivefalse
    \divis=#1
    \mod{\divis}{3}
    \ifnum \divis=0 \dthreetrue\fi
    \divis=#1
    \mod{\divis}{5}
    \ifnum \divis=0 \dfivetrue\fi
    \ifdthree{}Fizz\fi
    \ifdfive{}Buzz\fi
    \par
}

% \fizzbuzz{from}{to} Will do the fizzbuzz operation 
% from 'from' to 'to'
\def\fizzbuzz #1#2{
    \fstart=#1
    \fend=#2
    \divis=0
    \ifnum \fstart = \fend
        \empty%
    \else%
    \loop{}%
    \fb{\fstart}
    \ifnum \fstart < \fend
        \advance \fstart by 1
    \repeat
    \fi
}

% This is the actual command that generates the output
\fizzbuzz{1}{100}
\end
